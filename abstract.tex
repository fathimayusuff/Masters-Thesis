With the advancements in semantic web technologies, researchers have developed several question answering systems that interpret the user’s natural language question and fetches the answer from knowledge graphs. One of the main challenges of building question answering systems is determining which relations within a knowledge graph matches the keywords found in the Natural Language question. In order to bridge the gap between the simple yet ambiguous natural language question, and the difficult relation mapping problem, we propose ReMLOFT, an interactive relation mapping approach which relies on external evidence from a large corpus of text for mapping relations to the keywords found in a Natural Language question without using any training data. Our approach builds a free-text knowledge graph from Wikipedia, with entities as nodes and sentences in which these entities co-occur, as edges. ReMLOFT interactively helps the user choose better candidate relations to build fine-grained SPARQL queries. In addition, we build a dictionary of the most frequent keywords that define the context of a relation in the knowledge graph without using contemporary lexical tools. Experiments on three question answering datasets show our approach can map high quality candidate relations in comparison to statistical and embedding-based relation mapping approaches. 
