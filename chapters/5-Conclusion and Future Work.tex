In this thesis, we present an interactive relation mapping approach, ReMLOFT, for answering simple and complex questions, by leveraging an information-dense free-text knowledge graph. We also investigate the utilization of the free-text knowledge graph to find out which keywords are semantically similar to the ones in the question by creating a dictionary of the most frequent keywords using example sentences from Wikipedia. While keeping the model simple by refraining from using highly complicated machine learning models and neural networks, our approach achieves competitive results when compared to the baselines. 

In comparison to other relation mapping approaches \cite{falcon, falcon2, kbpearl}, we explore the relation mapping problem as an interactive method by exploiting richer word-level semantics than those captured by string similarity or word embedding techniques. Our approach gives the user the pliability to investigate various options to build better SPARQL queries for question answering systems. In addition, our approach could be integrated with existing query completion and query assisting tools to enhance the capability of their systems.

We believe our model is easily transferable to various knowledge graphs across multiple domains and is acceptable for frequently evolving KGs, as our approach does not learn KG-specific representations. 

In future, we aim to focus on better candidate pruning and noise reduction to improve the learning of relation patterns. We further plan to extend the assessment of the portability of our model across various domains and knowledge graphs. Our approach will be fine-tuned to produce more compelling outcomes, allowing for a more comprehensive quantitative comparison.
